\documentclass{article}
\usepackage[utf8]{inputenc}
\usepackage{amsmath}
\usepackage{mathtools}
\usepackage{amsthm}
\usepackage{graphicx}
\usepackage{tabularx}
\usepackage{mathtools}
\usepackage{mathrsfs}
\usepackage{enumerate}
\usepackage{amssymb}
\usepackage{accents}
\usepackage{commath}
\usepackage{yfonts}
\usepackage{float}
\usepackage{array}
\usepackage{tikz-cd} 

\title{Math3302 Assignment 1}
\author{Dominic Scocchera}
\date{March 2023}

\newtheorem{theorem}{Theorem}
\newtheorem{corollary}{Corollary}
\newtheorem{lemma}[theorem]{Lemma}

\begin{document}
\maketitle
\section*{Q1}
\subsection*{a)}
\subsection*{b)}
\section*{Q2}
\subsection*{a)}
We will use algorithm 4.3.1 to find a basis for the linear code $C=\langle S\rangle$.
\begin{align*}
A&=\begin{pmatrix}
2 &1&0&2&1&2\\
1&1&1&1&1&0\\
0&2&2&1&0&1\\
\end{pmatrix}\\
&\rightarrow \begin{pmatrix}
2 &1&0&2&1&2\\
1&0&0&2&1&1\\
0&2&2&1&0&1\\
\end{pmatrix},\;\;\;R_2=R_2+R_3\\
&\rightarrow \begin{pmatrix}
0 &1&0&1&2&0\\
1&0&0&2&1&1\\
0&2&2&1&0&1\\
\end{pmatrix},\;\;\;R_1=R_1+R_2\\
&\rightarrow \begin{pmatrix}
0 &1&0&1&2&0\\
1&0&0&2&1&1\\
0&0&2&2&2&1\\
\end{pmatrix},\;\;\;R_3=R_3+R_1\\
&\rightarrow \begin{pmatrix}
1&0&0&2&1&1\\
0&1&0&1&2&0\\
0&0&2&2&2&1\\
\end{pmatrix},\;\;\;R_1\leftrightarrow R_2\\
&\rightarrow \begin{pmatrix}
1&0&0&2&1&1\\
0&1&0&1&2&0\\
0&0&1&1&1&2\\
\end{pmatrix},\;\;\;R_3=R_3+R_3\\
\end{align*}
So a basis for C is thus $\{100211,010120,001112\}$ and hence the generating matrix is:
$$G_C=\begin{pmatrix}
1&0&0&2&1&1\\
0&1&0&1&2&0\\
0&0&1&1&1&2\\
\end{pmatrix}$$
\subsection*{b)}
Now from a) we see that as $G_C=(I\;\;\;X)$ that C is a systematic code and hence:
\begin{align*}
H_C&=\begin{pmatrix}
-X\\
I\\
\end{pmatrix}\\
&=\begin{pmatrix}
1&2&2\\
2&1&0\\
2&2&1\\
1&0&0\\
0&1&0\\
0&0&1\\
\end{pmatrix}
\end{align*}
\subsection*{c)}
$H_C$ has no rows of zeros so $\delta>1$. $H_C$ has no pair of identical rows so $\delta>2$. Rows 1, 3 and 5 sum to zero so $\delta=3$.
\subsection*{d)}
$$\begin{pmatrix}1&2&1\\\end{pmatrix}G_c=\begin{pmatrix}1&2&1&2&0&0\\\end{pmatrix}$$
\subsection*{e)}
$$\begin{pmatrix}0&1&1&0&2&2\end{pmatrix}H_C=\begin{pmatrix}1&2&0\end{pmatrix}$$
So the syndrome is $\begin{pmatrix}1&2&0\end{pmatrix}$. From $H_C$ we get that an SDA is as follows (note we are assuming there is at most a single error):
\begin{center}
\begin{tabular}{||c c||} 
 \hline
 Coset Leader &  Syndrome\\ [0.5ex] 
 \hline\hline
 000000 & 000\\ 
 \hline
 100000 & 122\\
 \hline
 010000 & 210\\
 \hline
 001000 & 221\\
 \hline
 000100 & 100\\
 \hline
 000010 & 010\\
 \hline
 000001 & 001\\
 \hline
 200000 & 211\\
 \hline
 020000 & 120\\
 \hline
 002000 & 112\\ 
 \hline
 000200 & 200\\
 \hline
 000020 & 020\\
 \hline
 000002 & 002\\  [1ex] 
 \hline
\end{tabular}
\end{center}
So the error is a 2 in position 2. So the most likely codeword is $011022-020000=021022$. Hence the codeword that was sent was $021$. 
\section*{Q3}
The Griesmer bound for a linear $(n,k,\delta)$ code is:
$$n\geq\sum_{j=0}^{k-1}\left\lceil\frac{\delta}{2^j}\right\rceil$$
We also have that a Reed-Muller code is a linear $(2^m,m+1,2^{m-1})$ code. Plugging these values into the Griesmer bound we get:
\begin{align*}
\sum_{j=0}^{k-1}\left\lceil\frac{\delta}{2^j}\right\rceil&=\sum_{j=0}^{(m+1)-1}\left\lceil\frac{2^{m-1}}{2^j}\right\rceil\\
&=\sum_{j=0}^{m}\left\lceil2^{m-j-1}\right\rceil\\
&=(2^{m-1}+2^{m-2}+...+2^{1}+2^{0})+\left(\left\lceil2^{-1}\right\rceil\right)\\
&=(2^m-1)+(1)\\
&=2^m\\
&=n\\
\end{align*}
So the Reed-Muller code achieves the Griesmer bound with equality.
\section*{Q4}
\subsection*{a)}
We first see that we are dealing with a binary linear (15,k,6) code. From the notes we have that the Hamming bound is:
$$k\leq 15-\left\lceil\log_2\left(\sum_{j=0}^{\left\lfloor\frac{6-1}{2}\right\rfloor}{15\choose j}\right)\right\rceil=8$$
For the Griesmer bound we have:
$$15\geq \sum_{j=0}^{k-1}\left\lceil\frac{6}{2^j}\right\rceil$$
We see that the RHS is an increasing function of k and the inequality only holds for $k\leq7$ (for k=7 the RHS equals 15). The Griesmer bound is better as it rules out the existance of codes of dimension 8 unlike the Hamming bound.
\subsection*{b)}
We know from class that a (23,12,7) code exists and that if an $(n,k,\delta)$ code exists then both an $(n-1,k-1,\delta)$ and a $(n-1,k,\delta-1)$ code exist. If we apply $(n-1,k-1,\delta)$ seven times and $(n-1,k,\delta-1)$ once to the (23,12,7) code we get a $(15,5,6)$ code. We also know that if a $(n,k,\delta)$ code exists then a $(n,j,\delta)$ code exists for $j\in\{1,...,k\}$. Hence there also exists a $(15,4,6)$, $(15,3,6)$, $(15,2,6)$ and a $(15,1,6)$ code. This is what we wanted to show.
\section*{Q5}
\section*{Q6}
\section*{Q7}
\subsection*{a)}
\subsection*{b)}
\end{document}