\documentclass{article}
\usepackage[utf8]{inputenc}
\usepackage{amsmath}
\usepackage{mathtools}
\usepackage{amsthm}
\usepackage{graphicx}
\usepackage{tabularx}
\usepackage{mathtools}
\usepackage{mathrsfs}
\usepackage{enumerate}
\usepackage{amssymb}
\usepackage{accents}
\usepackage{commath}
\usepackage{yfonts}
\usepackage{float}
\usepackage{array}
\usepackage{tikz-cd} 

\title{Math3302 Assignment 1}
\author{Dominic Scocchera, s4642675, Tute 1}
\date{April 2023}

\newtheorem{theorem}{Theorem}
\newtheorem{corollary}{Corollary}
\newtheorem{lemma}[theorem]{Lemma}

\begin{document}
\maketitle
\section*{Q1}
\subsection*{a)}
We first see that as $\left\lfloor\frac{\delta-1}{2}\right\rfloor=2$ we have $\delta=5$ or $\delta=6$. Now from the table we see that a (15,7,6) code is able to produce 128 message words which is the minimum for producing at least 120 message words for a (n,k,6) code. Using the result from the table $A_2(n-1,2e-1)=A_2(n,2e)$ we get that a $(14,7,5)$ code exists. 14 is the minimum n that satisfies what we need. We can't decrease it further as decreasing by 1 we get, $64=A_2(14,6)=A_2(13,5)$.
\subsection*{b)}
As our (14,7,6) code is a 2-error correcting code we transmit over a binary symmetric channel of reliability p = 0.9995, for each codeword we can correct 0,1 or 2 errors, thus giving:
$$Q(c)=p^{14}+{14 \choose 1}p^{14-1}(1-p)+{14 \choose 2}p^{14-2}(1-p)^2=0.999999954687$$
This is the same for each codeword so:
$$Q_C=0.999999954687$$
$$F_C=1-Q_C=4.5312686603\times10^{-8}<4.531\times10^{-8}$$
Which is what we wanted to show.
\section*{Q2}
\subsection*{a)}
We will use algorithm 4.3.1 to find a basis for the linear code $C=\langle S\rangle$.
\begin{align*}
A&=\begin{pmatrix}
2 &1&0&2&1&2\\
1&1&1&1&1&0\\
0&2&2&1&0&1\\
\end{pmatrix}\\
&\rightarrow \begin{pmatrix}
2 &1&0&2&1&2\\
1&0&0&2&1&1\\
0&2&2&1&0&1\\
\end{pmatrix},\;\;\;R_2=R_2+R_3\\
&\rightarrow \begin{pmatrix}
0 &1&0&1&2&0\\
1&0&0&2&1&1\\
0&2&2&1&0&1\\
\end{pmatrix},\;\;\;R_1=R_1+R_2\\
&\rightarrow \begin{pmatrix}
0 &1&0&1&2&0\\
1&0&0&2&1&1\\
0&0&2&2&2&1\\
\end{pmatrix},\;\;\;R_3=R_3+R_1\\
&\rightarrow \begin{pmatrix}
1&0&0&2&1&1\\
0&1&0&1&2&0\\
0&0&2&2&2&1\\
\end{pmatrix},\;\;\;R_1\leftrightarrow R_2\\
&\rightarrow \begin{pmatrix}
1&0&0&2&1&1\\
0&1&0&1&2&0\\
0&0&1&1&1&2\\
\end{pmatrix},\;\;\;R_3=R_3+R_3\\
\end{align*}
So a basis for C is thus $\{100211,010120,001112\}$ and hence the generating matrix is:
$$G_C=\begin{pmatrix}
1&0&0&2&1&1\\
0&1&0&1&2&0\\
0&0&1&1&1&2\\
\end{pmatrix}$$
\subsection*{b)}
Now from a) we see that as $G_C=(I\;\;\;X)$ that C is a systematic code and hence:
\begin{align*}
H_C&=\begin{pmatrix}
-X\\
I\\
\end{pmatrix}\\
&=\begin{pmatrix}
1&2&2\\
2&1&0\\
2&2&1\\
1&0&0\\
0&1&0\\
0&0&1\\
\end{pmatrix}
\end{align*}
\subsection*{c)}
$H_C$ has no rows of zeros so $\delta>1$. $H_C$ has no pair of identical rows so $\delta>2$. Rows 1, 3 and 5 sum to zero so $\delta=3$.
\subsection*{d)}
$$\begin{pmatrix}1&2&1\\\end{pmatrix}G_c=\begin{pmatrix}1&2&1&2&0&0\\\end{pmatrix}$$
\subsection*{e)}
$$\begin{pmatrix}0&1&1&0&2&2\end{pmatrix}H_C=\begin{pmatrix}1&2&0\end{pmatrix}$$
So the syndrome is $\begin{pmatrix}1&2&0\end{pmatrix}$. From $H_C$ we get that an SDA is as follows (note we are assuming there is at most a single error):
\begin{center}
\begin{tabular}{||c c||} 
 \hline
 Coset Leader &  Syndrome\\ [0.5ex] 
 \hline\hline
 000000 & 000\\ 
 \hline
 100000 & 122\\
 \hline
 010000 & 210\\
 \hline
 001000 & 221\\
 \hline
 000100 & 100\\
 \hline
 000010 & 010\\
 \hline
 000001 & 001\\
 \hline
 200000 & 211\\
 \hline
 020000 & 120\\
 \hline
 002000 & 112\\ 
 \hline
 000200 & 200\\
 \hline
 000020 & 020\\
 \hline
 000002 & 002\\  [1ex] 
 \hline
\end{tabular}
\end{center}
So the error is a 2 in position 2. So the most likely codeword is $011022-020000=021022$. Hence the codeword that was sent was $021$. 
\section*{Q3}
The Griesmer bound for a linear $(n,k,\delta)$ code is:
$$n\geq\sum_{j=0}^{k-1}\left\lceil\frac{\delta}{2^j}\right\rceil$$
We also have that a Reed-Muller code is a linear $(2^m,m+1,2^{m-1})$ code. Plugging these values into the Griesmer bound we get:
\begin{align*}
\sum_{j=0}^{k-1}\left\lceil\frac{\delta}{2^j}\right\rceil&=\sum_{j=0}^{(m+1)-1}\left\lceil\frac{2^{m-1}}{2^j}\right\rceil\\
&=\sum_{j=0}^{m}\left\lceil2^{m-j-1}\right\rceil\\
&=(2^{m-1}+2^{m-2}+...+2^{1}+2^{0})+\left(\left\lceil2^{-1}\right\rceil\right)\\
&=(2^m-1)+(1)\\
&=2^m\\
&=n\\
\end{align*}
So the Reed-Muller code achieves the Griesmer bound with equality.
\section*{Q4}
\subsection*{a)}
We first see that we are dealing with a binary linear (15,k,6) code. From the notes we have that the Hamming bound is:
$$k\leq 15-\left\lceil\log_2\left(\sum_{j=0}^{\left\lfloor\frac{6-1}{2}\right\rfloor}{15\choose j}\right)\right\rceil=8$$
For the Griesmer bound we have:
$$15\geq \sum_{j=0}^{k-1}\left\lceil\frac{6}{2^j}\right\rceil$$
We see that the RHS is an increasing function of k and the inequality only holds for $k\leq7$ (for k=7 the RHS equals 15). The Griesmer bound is better as it rules out the existance of codes of dimension 8 unlike the Hamming bound.
\subsection*{b)}
We know from class that a (23,12,7) code exists and that if an $(n,k,\delta)$ code exists then both an $(n-1,k-1,\delta)$ and a $(n-1,k,\delta-1)$ code exist. If we apply $(n-1,k-1,\delta)$ seven times and $(n-1,k,\delta-1)$ once to the (23,12,7) code we get a $(15,5,6)$ code. We also know that if a $(n,k,\delta)$ code exists then a $(n,j,\delta)$ code exists for $j\in\{1,...,k\}$. Hence there also exists a $(15,4,6)$, $(15,3,6)$, $(15,2,6)$ and a $(15,1,6)$ code. This is what we wanted to show.
\section*{Q5}
We want to show that for each integer $s\geq4$, there exists a linear binary code of length 2s+1, dimension 2s and distance 8. I couldn't figure out how to construct s=4 so here is a proof for $s\geq5$.
\begin{proof}
We begin by showing that the GV bound gives the existance for s=5.
$${64-1\choose0}+...+{64-1\choose8-2}=75611761<4294967296=2^{64-32}$$
We will now proceed by induction with our base case being the construction of $s=6$. We will make use of the theorem that states that if we let $C_1$ be a linear $(n, k_1, \delta_1)$-code and let $C_2$ be a linear $(n, k_2, \delta_2)$-code. Then there exists a linear $(2n, k_1 + k_2, d)$-code where $d = \min\{2\delta_1, \delta_2\}$. For our base case we let $C_1=C_2=(64,32,8)$, which is exactly the s=5 code and by the above theorem we get that there exists a $(2\cdot64,32+32,\min\{2\cdot8, 8\})=(128,64,8)$ which is the s=6 code. Now we assume that it holds for $m$ and show that it also holds for $m+1$. By our inductive hypothesis we have that the $(2^{m+1},2^m,8)$ code exists and we want to show $(2^{m+2},2^{m+1},8)$ code exists. Taking $C_1=C_2=(2^{m+1},2^m,8)$, by the above theorem we get that $(2\cdot2^{m+1},2^{m}+2^{m},\min\{2\cdot8, 8\})=(2^{m+2},2^{m+1},8)$ code exists. Hence by the principle of mathematical induction we have shown that there always exists codes of the form $(2^{s+1},2^s,8)$ for all $s\geq4$. 
\end{proof}
\section*{Q6}
Our word is $w=10010101100111100010000$. We have $\lVert w \rVert=10$, we require $\lVert w*\rVert$ to be odd, so $w*=100101011001111000100001$. We now calculate the syndrome, $s=w*H=w*\begin{pmatrix}I_{12}\\B\\\end{pmatrix}=011010010001$. Now computing the sum of s and $b_j$ (each row of B) we get:

\begin{center}
\begin{tabular}{||c c||}
 \hline
 $s+b_j$&$\lVert s+b_j\rVert$\\ [0.5ex] 
 \hline
 101101010100100000000000 & 7 \\
 \hline
 110100011010110000000000 & 8\\ 
 \hline
 000110000110111000000000 & 7 \\
 \hline
 100010111100111100000000 & 10 \\
 \hline
 101011001010111110000000 & 11\\
 \hline
 111000100110111111000000 & 12 \\
 \hline
 011111111110111111100000 & 17\\
 \hline
 010001001100111111110000 & 12\\
 \hline
 001100101000111111111000 & 13\\
 \hline
 110111100000111111111100 & 16\\
 \hline
 000001110010111111111110 & 15\\
 \hline
 100101101111111111111111 & 20\\[1ex] 
 \hline 
\end{tabular}
\end{center}
None of these are less than 2 so we continue. Now we comput the second syndrome, $s'=sB=110111100100$. We have $\lVert s'\rVert=7>3$ so we continue. Now computing the sum of s' and $b_1$ we get $s'+b_1=000000100001$, and as $\lVert s'+b_1\rVert=2$ we have $e=(\theta_1|s'+b_1)=100000000000000000100001$. Finally decoding we get $c*=w*+e=000101011001111000000000$. Now we remove the last bit to get that our codeword is $c=00010101100111100000000$.
\section*{Q7}
\subsection*{a)}
We have that our recieved word is $w=10110 10000 10111$, $g(x)=1+x+x^2+x^3+x^6$. This generates a 3 cyclic burst error correcting linear code. We assume that a cyclic burst error pattern of burst length at most 3 has occured. From example 7.6.3 we know:
$$H =\begin{pmatrix}
1& 1& 1& 1& 0& 0\\
0& 1& 1& 1& 1& 0\\
0& 0& 1& 1& 1& 1\\
1& 1& 1& 0& 1& 1\\
1& 0& 0& 0& 0& 1\\
1& 0& 1& 1& 0& 0\\
0& 1& 0& 1& 1& 0\\
0& 0& 1& 0& 1& 1\\
1& 1& 1& 0& 0& 1\\
1&0&0&0&0&0\\
0&1&0&0&0&0\\
0&0&1&0&0&0\\
0&0&0&1&0&0\\
0&0&0&0&1&0\\
0&0&0&0&0&1\\
\end{pmatrix}$$
We now calculate the syndrome, $s=wH=110011$. This corresponds to the polynomial $x^5+x^4+x+1$ Now we calculate the shifted syndromes until burst length is $\leq3$.
\begin{center}
\begin{tabular}{||c c||} 
 \hline
 $s_i=x^i\cdot s \;\;\text{mod }g(x)$ & burst length\\ [0.5ex] 
 \hline\hline
 $1+x^3+x^5$ & 6\\ 
 \hline
 $1+x^2+x^3+x^4$ & 5\\
 \hline
 $x(1+x^2+x^3+x^4)$ & 5\\
 \hline
 $1+x+x^3+x^4+x^5$ & 6 \\
 \hline
 $1+x^3+x^4+x^5$ & 6\\
 \hline
 $1+x^2+x^3+x^4+x^5$ & 6\\
 \hline
 $1+x^2+x^4+x^5$ & 6\\
 \hline
 $1+x^2+x^5$ & 6\\
 \hline
 $1+x^2$ & 3\\ [1ex] 
 \hline
\end{tabular}
\end{center}
$s_9$ is the only shifted syndrome with burst length $\leq3$ so:
\begin{align*}
e(x)&\equiv x^{9-9}s_9(x)\;\;\;\text{mod }1+x^{15}\\
&=1+x^2\;\;\;\text{mod }1+x^{15}\\
&=101000000000000\\
\end{align*}
So now we get our codeword is:
$$c=w+e=000101000010111$$
\subsection*{b)}
Given a word $w(x)$ we encode it by performing the operation $g(x)w(x)=c(x)$.  So from a) we get:
\begin{align*}
g(x)w(x)&=c(x)\\
\iff(1+x+x^2+x^3+x^6)w(x)&=x^3+x^5+x^10+x^12+x^13+x^14\\
\iff w(x)&=\frac{x^3+x^5+x^10+x^12+x^13+x^14}{1+x+x^2+x^3+x^6}\\
\iff w(x)&=x^3+x^4+x^5+x^6+x^7+x^8\\
&=000111111000000\\
\end{align*}
\end{document}