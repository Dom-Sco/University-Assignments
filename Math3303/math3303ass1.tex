\documentclass{article}
\usepackage[utf8]{inputenc}
\usepackage{amsmath}
\usepackage{mathtools}
\usepackage{amsthm}
\usepackage{graphicx}
\usepackage{tabularx}
\usepackage{mathtools}
\usepackage{mathrsfs}
\usepackage{enumerate}
\usepackage{amssymb}
\usepackage{accents}
\usepackage{commath}
\usepackage{yfonts}
\usepackage{float}
\usepackage{array}

\title{Math3303 Assignment 1}
\author{Dominic Scocchera}
\date{Febuary 2023}

\newtheorem{theorem}{Theorem}
\newtheorem{corollary}{Corollary}
\newtheorem{lemma}[theorem]{Lemma}

\begin{document}
\maketitle
\section*{Q1}
Let $\text{G} = \text{GL}_n(\mathbb{R})$ be the group of $n \times n$ invertible matrices and $\text{N} = \text{SL}_n(\mathbb{R})$ the subgroup of G consisting of those matrices which have determinant one. First we want to prove that N$\unlhd$G.
\begin{proof}
By definition we have that, N$\unlhd$G$\iff\forall g\in$G and $n\in$N, $gng^{-1}\in$N. This means we require det($gng^{-1}$)=1 $\forall$ g$\in$G and $n\in$N, as N is the group of invertible matrices whose determinant is 1. By calculation and properties of determinant ($\det(AB)=\det(A)\det(B)$, $\det(A^{-1})=\frac{1}{\det(A)}$, $\forall n\in\text{N},\;\;\det(n)=1$) we get:
\begin{align*}
    \det(gng^{-1})&=\det(g)\det(n)\det(g^{-1})\\
    &=\frac{\det(g)}{\det(g)}\det(n)\\
    &=1\\
\end{align*}
\end{proof}
\noindent Now we want to prove $\text{G}/\text{N}\cong\mathbb{R}^*$.
\begin{proof}
First we define the homomorphism $\phi:\text{GL}_n(\mathbb{R})\rightarrow\mathbb{R}^*$. This is a homomorphism because $\forall A,B\in\text{GL}_n(\mathbb{R})$, $\phi(A\cdot B)=\det(A\cdot B)=\det(A)\cdot\det(B)\in\mathbb{R}^*$. As the identity element of $\mathbb{R}^*$ is 1 we have that:
$$\text{Ker}\;\phi=\{A\in\text{GL}_n(\mathbb{R}):\det(A)=1\}=\text{SL}_n(\mathbb{R})$$
Now by the first isomorphism theorem we have:
\begin{align*}
\text{GL}_n(\mathbb{R})/\text{Ker}\;\phi&=\text{GL}_n(\mathbb{R})/\text{SL}_n(\mathbb{R})\\
&=G/N\\
&\cong\phi(G)\\
\end{align*}
Now to get the result we require $\phi(G)\cong\mathbb{R}^*$ which occurs if the homorphism is surjective, i.e. $\forall a\in\mathbb{R}^*,\;\exists A\in \text{G } \text{s.t. } \phi(A)=\det(A)=a$. To show this we consider the matrix whose top left value is a, the rest of the diagonal is 1 and every other entry is 0. The determinant of this matrix is clearly a, and so we have a surjective homomorphism, hence the result.
\end{proof}
\section*{Q2}
Let $G = \text{SL}_2(\mathbb{Z})$ be the group of $2 \times 2$ matrices with integer coefficients and determinant equal to 1. We want to show that $G$ is generated by:
$$S:=\begin{pmatrix}
0 & -1\\
1 & 0\\
\end{pmatrix}, \;\;\;\;T:=\begin{pmatrix}
1 & 1\\
0 & 1\\
\end{pmatrix}$$
\begin{proof}
We first note that $S^4=I$ where $I$ is the identity matrix and $T^n=\begin{pmatrix}
1 & n\\
0 & 1\\
\end{pmatrix}$.
We first denote the subgroup of $\text{SL}_2(\mathbb{Z})$ generated by S and T as $G$. We now note that:
$$SA=\begin{pmatrix}
-c & -d\\
a & b\\
\end{pmatrix}, \;\;\;\;T^nA=\begin{pmatrix}
a+nd & b+nd\\
c & d\\
\end{pmatrix}$$
Where $A=\begin{pmatrix}
a & b\\
c & d\\
\end{pmatrix}$. Now choose $A$ s.t. $A\in\text{SL}_2(\mathbb{Z})$ and suppose $c\neq0$. Now consider when $|a| \geq |c|$. If this is the case then we divide a by c yeilding $a = cp + q$ with $0 \leq q < |c|$. Applying the note from above we get that $T^{-q} A$ has $a - pc = q$ in its upper left corner. This is smaller in absolute value than the lower left entry c in $T^{-q}A$. As we saw above multiplying by S switches the top and bottom entries, with the top entries changing sign. Now we can apply the division algorithm in $\mathbb{Z}$ again if the lower left entry is nonzero in order to find another power of $T$ to multiply by on the left so the lower left entry has smaller absolute value than before. Eventually multiplication of $A$ on the left by enough copies of $S$ and powers of $T$ gives a matrix in $\text{SL}_2 (\mathbb{Z})$ with lower left entry 0. Such a matrix, m since it is integral with determinant 1, has the form $\begin{pmatrix} 
\pm 1 & m\\
0 & \pm 1\\
\end{pmatrix}$ for some $m \in \mathbb{Z}$ and common $m$ $-m$ signs on the diagonal. This matrix is either $T$ or $-T$, so there is some $g\in G$ such that $gA = \pm T^n$ for some $n\in\mathbb{Z}$. Since $T^n\in G$ and $S^2=-I_2$, we have $A=\pm g^{-1}T^n\in G$.
\end{proof}
\section*{Q3}
Let $U$ denote the set of roots of unity in $\mathbb{C}^*$. That is,
$$U:=\{x\in\mathbb{C}\;|\;x^n=1,\;\;\text{for some }n\in\mathbb{Z}_{\geq0}\}$$
We want to show that $\mathbb{Q}/\mathbb{Z}\cong U$.
\begin{proof}
We begin by considering the homomorphism $\phi:x\rightarrow e^{2\pi ix}, \;\;x\in\mathbb{Q}$. This is a homomorphism because if we consider $x,y\in\mathbb{Q}$ we get:
\begin{align*}
\phi(x+y)&=e^{2\pi i(x+y)}\\
&=e^{2\pi ix}e^{2\pi iy}\\
&=\phi(x)\cdot\phi(y)\\
\end{align*}
Now if we restrict x to the integers ($x\in\mathbb{Z}$) and apply Eulers identity and the fact that $\cos(2\pi x)=1$ and $\sin(2\pi x)=0$ $\forall x\in\mathbb{Z}$ we get:
\begin{align*}
e^{2\pi ix}&=\cos(2\pi x)+i\sin(2\pi x)\\
&=1+i\cdot0\\
&=1\\
\end{align*}
As 1 is the multiplicative identity in $\mathbb{C}^*$ we have that $\text{Ker }\phi=\mathbb{Z}$. Now by the first isomorphism theorem we have $\mathbb{Q}/\mathbb{Z}\cong \phi(\mathbb{Q})$. To get the result we must show that the homomorphism is surjective as that implies $\phi(\mathbb{Q})\cong U$. This means that $\forall z\in U,\;\;\exists a\in\mathbb{Q}$ s.t. $\phi\left(a\right)=z$. We have that $\phi(a)=e^{2\pi i a}=\cos(2\pi a)+i\sin(2\pi a)$ which is exactly the roots of unity as for $a=n$ this is one. This means the homomorphism is surjective, and hence the result follows.
\end{proof}
\end{document}