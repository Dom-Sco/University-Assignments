\documentclass{article}
\usepackage[utf8]{inputenc}
\usepackage{amsmath}
\usepackage{mathtools}
\usepackage{amsthm}
\usepackage{graphicx}
\usepackage{tabularx}
\usepackage{mathtools}
\usepackage{mathrsfs}
\usepackage{enumerate}
\usepackage{amssymb}
\usepackage{accents}
\usepackage{commath}
\usepackage{yfonts}
\usepackage{float}
\usepackage{array}
\usepackage{tikz-cd} 

\title{Math3303 Assignment 4}
\author{Dominic Scocchera}
\date{March 2023}

\newtheorem{theorem}{Theorem}
\newtheorem{corollary}{Corollary}
\newtheorem{lemma}[theorem]{Lemma}

\begin{document}
\maketitle
\section*{Q1}
Consider the action of $GL_2(\mathbb{Z})$ on $\mathbb{Z}^2$. Determine all the orbits and stabilisers of this action.
\newline\newline
We first note that for a $2\times2$ matrix $A=\begin{pmatrix}
a & b\\
c & d\\
\end{pmatrix}$, it's inverse is $A^{-1}=\frac{1}{\text{det}(A)}\begin{pmatrix}
d & -b\\
-c & a\\
\end{pmatrix}$ and as the inverses must also be in the group the entries remain integers if and only if $\text{det}(A)=ad-bc=\pm 1$. We also note that the orbit and stabiliser of the zero vector are:
\begin{align*}
\mathcal{O}(\mathbf{0})&=\{\mathbf{0}\}\\
\text{Stab}(\mathbf{0})&=GL_2(\mathbb{Z})\\
\end{align*}
Now we want to show that the orbits in $\mathbb{Z}^2$ under the action of $GL_2(\mathbb{Z})$ are the vectors whose coordinates have a fixed greatest common divisor. Each orbit contains one vector of the form $\begin{pmatrix}
m\\
0\\
\end{pmatrix}$ for $m\geq 0$, and the stabiliser of $\begin{pmatrix}
m\\
0\\
\end{pmatrix}$ for $m > 0$ is $\left\{\begin{pmatrix}
1 & x\\
0 & y\\
\end{pmatrix} |\;\;\; y = \pm1,x\in\mathbb{Z}\right\} \subset GL_2(\mathbb{Z})$.
\begin{proof}
We first note that $\text{gcd}(m,0)=m$, so the fixed gcd in each orbit is going to be $m$. We also note that the stabiliser is trivially true as 
$\begin{pmatrix}
1 & x\\
0 & y\\
\end{pmatrix}\begin{pmatrix}
m\\
0\\
\end{pmatrix}=\begin{pmatrix}
m\\
0\\
\end{pmatrix}$. We also have
that $\begin{pmatrix}
a & b\\
c & d\\
\end{pmatrix}\begin{pmatrix}
m\\
0\\
\end{pmatrix}=\begin{pmatrix}
ma\\
mc\\
\end{pmatrix}$. We have that a and c are relatively prime as $ad-bc=\pm 1$, which means that $\text{gcd}(ma,mc)=m$. We now see that each vector of the form $\begin{pmatrix}
g\\
h\\
\end{pmatrix}$, where $\text{gcd}(g,h)=m$ is in the orbit of $\begin{pmatrix}
m\\
0\\
\end{pmatrix}$. We can solve $gx+hy=m$ for some integers x and y so $\begin{pmatrix}
\frac{g}{m} & -y\\
\frac{h}{m} & x\\
\end{pmatrix}$ is in $GL_2(\mathbb{Z})$ and from the solution to $gx+hy=m$, we get that the determinant is $\frac{g}{m}x+\frac{h}{m}y=1$. Also note that these fractions are in the integers as $\text{gcd}(g,h)=m$, so m is a divisor of both $g$ and $h$. Finally $\begin{pmatrix}
\frac{g}{m} & -y\\
\frac{h}{m} & x\\
\end{pmatrix}\begin{pmatrix}
m\\
0\\
\end{pmatrix}=\begin{pmatrix}
g\\
h\\
\end{pmatrix}$. The stabiliser for all elements of the form $\begin{pmatrix}
g\\
h\\
\end{pmatrix}$ is $\left\{\begin{pmatrix}
1&0\\
0&1\\
\end{pmatrix}\right\}$.
\end{proof}
\section*{Q2}
Let $G$ be a finite group acting transitively on a set $X$ satisfying $1 < |X| < \infty$. Show that there exists $g\in G$ which fixes no element of $X$.
\begin{proof}
As G is transitive (X has only one orbit) we get from the orbit stabiliser theorem, $|\text{Stab}(x)|=\frac{|G|}{|X|}$, $\forall x\in X$ (note that $|\text{Orb}(x)|=|X|$ by transistivity and the orbit stabiliser theorem states $|\text{Orb}(x)|=\frac{|G|}{|\text{stab}(x)|}$). Noting that every stabiliser contains the identity, we get:
\begin{align*}
\left|\bigcup_{x\in X}\text{Stab}(x)\right|&=\left|\bigcup_{x\in X}\{g\in G|gx=x\}\right|\\
&\leq 1+|G|-|X|\\
&=1+|X||\text{Stab}(x)|-|X|\\
&=1+|X|\left(|\text{Stab}(x)|-1|\right)\\
&=1+|X|\left(\frac{|G|}{|X|}-1\right)\\
&<|G|\\
\end{align*}
Hence $\bigcup_{x\in X}\text{Stab}(x)\neq G$ so $\exists g\in G$ s.t. $gx\neq x,\;\;\;\forall x\in X$. We also note the last line holds as:
\begin{align*}
1+|X|\left(\frac{|G|}{|X|}-1\right)&<|G|\\
\iff |G|-|X|&<|G|-1\\
\end{align*}
Which holds as $1<|X|$.
\end{proof}
\end{document}