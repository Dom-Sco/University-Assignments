\documentclass{article}
\usepackage[utf8]{inputenc}
\usepackage{amsmath}
\usepackage{mathtools}
\usepackage{amsthm}
\usepackage{graphicx}
\usepackage{tabularx}
\usepackage{mathtools}
\usepackage{mathrsfs}
\usepackage{enumerate}
\usepackage{amssymb}
\usepackage{accents}
\usepackage{commath}
\usepackage{yfonts}
\usepackage{float}
\usepackage{array}
\usepackage{tikz-cd} 

\title{Math3303 Assignment 3}
\author{Dominic Scocchera}
\date{March 2023}

\newtheorem{theorem}{Theorem}
\newtheorem{corollary}{Corollary}
\newtheorem{lemma}[theorem]{Lemma}

\begin{document}
\maketitle
\section*{Q1}
\subsection*{a)}
We want to show $G^{ab}$ is abelian.
\begin{proof}
In the previous assignment we showed that $[G,G]$ is a normal subgroup, so $G/[G,G]=\{g[G,G]:g\in G\}$. Hence we now get, $(g_1[G,G])(g_2[G,G])=(g_1 g_2)[G,G]=([g_1,g_2]g_2g_1)[G,G]=g_2g_1[G,G]=(g_2[G,G])(g_1[G,G])$ where $g_1,g_2\in G$.
\end{proof}
\subsection*{b)}
First we want to prove the fundamental homomorphism theorem.
\begin{theorem}
Let $G,H$ be groups, $f:G\rightarrow H$ a homomorphism, and let $N$ be a normal subgroup of $G$ such that $N\subseteq\text{ker }f$. Then there exists a unique homomorphism $f':G/N\rightarrow H$ so that $f'\circ\pi=f$, where $\pi$ denotes the obvious homomorphism from $G$ to $G/N$, $\pi(g)=gN$.
\end{theorem}
\begin{proof}
We first show the uniqueness of the mapping. Let $f_1',f_2':G/N\rightarrow H$ be functions such that $f_1'\circ\pi=f_2'\circ\pi$. For $y\in G/N$ there exists $x\in G$ such that $\pi(x)=y$, so we have $f_1'(y)=(f_1'\circ\pi)(x)=(f_2'\circ\pi)(x)=f_2'(y)$ for all $y\in G/N$, thus $f_1'=f_2'$. Now we define $f':G/N\rightarrow H,\;\;\;f'(gN)=f(g)\forall g\in G$. Now let $gN=kN$, or $k\in gN$. Since $N\subseteq \text{Ker }f$, $g^{-1}k\in N$ implies $g^{-1}k\in \text{Ker }f$, hence $f(g)=f(k)$. Clearly $f'\circ\pi=f$.
\end{proof}
\noindent Now we want to show that if we have the homomorphisms $\pi:G\rightarrow G^{ab}$, $f:G\rightarrow A$, with A abelian then there exists a homomorphism $f':G^{ab}\rightarrow A$ such that $f=f'\circ\pi$.
\begin{proof}
This is almost a direct consequence of the above theorem. As $f$ is mapping to an abelian group $A$ we have for $[a,b]\in[G,G]$:
\begin{align*}
f([a,b])&=f(aba^{-1}b^{-1})\\
&=f(a)f(b)f(a^{-1})f(b^{-1})\\
&=f(a)f(a^{-1})f(b)f(b^{-1})\\
&=f(aa^{-1}bb^{-1})\\
&=f(e)\\
&=e\\
\end{align*}
So we have $[G,G]\subseteq\text{Ker } f$. Plugging our values into the fundamental homomorphism theorem we get the desired result (note here that $N=[G,G]$ and $H=A$).
\end{proof}
\subsection*{c)}
Now we want to prove $G^{\vee}\cong(G^{ab})^{\vee}$.
\begin{proof}
Plugging the groups into the result from b) we get:
\newline
\begin{center}
\begin{tikzcd}
G \arrow[d, "\pi"] \arrow[r, "\phi"] & \mathbb{C}^{\times} \\
G^{ab} \arrow[ru, "\varphi"]
\end{tikzcd}
\end{center}
We note that we can do this as $[G,G]$ is a normal subgroup and $\mathbb{C}^{\times}$ is an abelian group. We also note that from b) we get $\phi=\varphi\circ\pi$ and that the homomorphisms are $\pi(g)=g[G,G]$ and $\varphi(\pi(g))=\phi(g)$ (From b)). Now consider $\theta:G^{\vee}\rightarrow(G^{ab})^{\vee}$, where $\theta(\phi(g))=\varphi(\pi(g))$. This is a homomorphism as:
\begin{align*}
\theta(\phi_1(g_1)\phi_2(g_2))&=\varphi(\pi(g_1g_2))\\
&=\varphi(\pi(g_1)\pi(g_2))\\
&=\varphi(\pi(g_1))\varphi(\pi(g_2))\\
&=\theta(\phi_1(g_1))\theta(\phi_1(g_2))\\
\end{align*}
Noting lines 2 and 3 are possible as $\phi$ and $\pi$ are homomorphisms. It is also bijective as $\phi(g)=\varphi(\pi(g))$, $\forall g\in G$, hence the result $G^{\vee}\cong(G^{ab})^{\vee}$.
\end{proof}
\section*{Q2}
We want to prove that $S_n$ can be generated by $(12)$ and $(12 ... n)$.
\begin{proof}
First we have:
\begin{align*}
(12...n)(1,2)(12...n)^{-1}&=(2,3)\\
(12...n)(2,3)(12...n)^{-1}&=(3,4)\\
&\vdots\\
(12...n)(n-2,n-1)(12...n)^{-1}&=(n-1,n)\\
\end{align*}
Hence $(i,i+1)\in\langle(1,2),(12...n)\rangle,\;\;\;\text{for all}\;\;1\leq i\leq n-1$. Next we have:
\begin{align*}
(2,3)(1,2)(2,3)^{-1}&=(1,3)\\
(3,4)(2,3)(3,4)^{-1}&=(1,4)\\
&\vdots\\
(n-1,n)(1,n-1)(n-1,n)^{-1}&=(1,n)\\
\end{align*}
From this we now have $(1,i)\in\langle (1,2),(12...n)\rangle$ for all $1\leq i\leq n$. Now for any $1\leq i<j\leq n$ we get the following:
$$(i,j)=(1,i)(1,j)(1,j)^{-1}\in\langle(1,2),(12...n)\rangle$$
Hence all transpositions are contained in $\langle(1,2),(12...n)\rangle$ and hence $\langle(1,2),(12...n)\rangle=S_n$ 
\end{proof}
\section*{Q3}
We want to show that $Z(S_n)=\{e\}$. We will first assume that $n\geq3$ as this trivially holds for $n=1$ and $n=2$.
\begin{proof}
By definition of identity in a group $(\forall g\in G,\;\;\; g\cdot e=e\cdot g)$ we have that $e\in Z(S_n)$. Also by the definition of center we have $Z(S_n)=\{\tau\in S_n:\forall\sigma\in S_n:\tau\sigma=\sigma\tau\}$. Let $\pi,\rho\in S_n$ be permutations of $\{1,...,n\}$. Suppose we have $\pi\in S_n$ such that $\pi\neq e,\pi(i)=j,i\neq j$. Since $n\geq3$, we can find $\rho\in S_n$ which interchanges j and k (where $k\neq i,j$) and fixes everything else. It follows that $\rho-1$ does the same thing, and in particular both $\rho$ and $\rho-1$ fix i. So: 
\begin{align*}
\rho\pi\rho^{-1}(i)&=\rho\pi(i)\\
&=\rho(j)\\
&=k\\
\end{align*}
So:
$$\rho\pi\rho^{-1}(i)=k\neq j=\pi(i)$$
Now we notice that if $\rho$ and $\pi$ were to commute, $\rho\pi\rho^{-1}=\pi$, but in $S_n$ this isn't the case. So for any $\pi\in S_n$ we can always find a $\rho$ such that $\rho\pi\rho^{-1}\neq\pi$. So no we have that no elements other than the identity of $S_n$ commute with all elements of $S_n$. Hence the result, $Z(S_n)=\{e\}$.
\end{proof}
\end{document}