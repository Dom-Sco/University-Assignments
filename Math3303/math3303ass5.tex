\documentclass{article}
\usepackage[utf8]{inputenc}
\usepackage{amsmath}
\usepackage{mathtools}
\usepackage{amsthm}
\usepackage{graphicx}
\usepackage{tabularx}
\usepackage{mathtools}
\usepackage{mathrsfs}
\usepackage{enumerate}
\usepackage{amssymb}
\usepackage{accents}
\usepackage{commath}
\usepackage{yfonts}
\usepackage{float}
\usepackage{array}
\usepackage{tikz-cd} 

\title{Math3303 Assignment 5}
\author{Dominic Scocchera}
\date{April 2023}

\newtheorem{lemma}{Lemma}
\newtheorem{theorem}{Theorem}

\theoremstyle{definition}
\newtheorem{definition}{Definition}[section]

\begin{document}
\maketitle
\section*{Q1}
We want to find all left, right and two-sided ideals of the ring of $n \times n$ complex matrices ($M_n(\mathbb{C})$). First we note that we have $1\leq i\leq n$ and $1\leq j\leq n$.
\newline\newline
Left Ideals:
\newline
Let $I_L$ be the set of left ideals, then we have:
\begin{align*}
1.\;\;&(I_L,+)\leq(M_n(\mathbb{C}),+)\\
2.\;\;&\forall A\in M_n(\mathbb{C})\text{ and } B\in I_L,\;\; AB\in I_L\\
\end{align*}
So $I_L$ is the set of matrices whose $j_{th}$ column is all zeros. If we have $AB=C$ then the entry $c_{ij}=\sum_{k=1}^{n}a_{ik}b_{kj}=0$ as all $b_{kj}=0$, and so the $j_{th}$ column of $C$ is all zeros and hence it is in $I_L$. It is also satisfies condition 1 as adding two matrices whose $j_{th}$ columns are zero gives another matrix whose $j_{th}$ column is zeros, hence it is closed. Also we note that the identity is just the zero matrix.
\newline\newline
Right Ideals:
\newline
Let $I_R$ be the set of right ideals, then we have:
\begin{align*}
1.\;\;&(I_R,+)\leq(M_n(\mathbb{C}),+)\\
2.\;\;&\forall A\in M_n(\mathbb{C})\text{ and } B\in I_L,\;\; BA\in I_R\\
\end{align*}
So $I_R$ is the set of matrices whose $i_{th}$ row is all zeros. If we have $BA=C$ then the entry $c_{ij}=\sum_{k=1}^{n}b_{ik}a_{kj}=0$ as all $b_{ik}=0$, and so the $i_{th}$ row of $C$ is all zeros and hence it is in $I_R$. It is also satisfies condition 1 as adding two matrices whose $i_{th}$ rows are zero gives another matrix whose $i_{th}$ row is zeros, hence it is closed. Also we note that the identity is just the zero matrix.
\newline\newline
Two Sided Ideals:
\newline
The two sided ideals are the set of matrices whose $i_{th}$ row and $j_{th}$ column are all zeros. It is clear to see this from above as such a matrix is a left sided ideal ($j_{th}$ column is all zeros) and also a right sided ideal ($i_{th}$ row is all zeros). As this set is both a left and right sided ideal it is a two sided ideal.
\section*{Q2}
We want to prove that $\mathbb{R}[x]/(x^2 + 1)$ is isomorphic to $\mathbb{C}$. First we will prove three lemmas:
\begin{lemma}
$x^2+1\in\mathbb{R}[x]$ is irreducible.
\end{lemma}
\begin{proof}
Suppose $x^2+1$ is reducible, then:
$$x^2+1=(ax+b)(cx+d)=acx^2+(ad+bc)x+bd$$
Hence we have that $ac=1$, $bd=1$ and $ad+bc=0$. This yields:
\begin{align*}
ad&=-bc\\
\implies (ac)d&=-bc^2\\
\implies d&=-bc^2\\
\implies (bd)&=-(bc)^2\\
\implies 1&=-(bc)^2\\
\end{align*}
Which isn't possible as this requires that $bc=i$ but this can only occur if $b$ and/or $c$ are complex numbers but they are real so $x^2+1$ is irreducible.
\end{proof}
\begin{definition}[Prinicipal Ideal Domain (PID)]
A principal ideal domain is an integral domain R in which every ideal has the form:
$$\langle a \rangle = \{ar|r \in R\} = aR$$
for some a in R.
\end{definition}
\begin{lemma}
Let R be a PID, then every non-zero prime ideal is maximal.
\end{lemma}
\begin{proof}
Let $P\subset R$ be a nonzero prime ideal. From the fact that $R$ is a PID we get that $P=\langle p\rangle$ for some $p\in R$. Now suppose there is an ideal $I= \langle x\rangle$ such that $\langle p\rangle\subseteq \langle x\rangle\subseteq R$. From this we have that $p\in \langle x\rangle$ so that $p=kx$ for some $k\in R$. As $\langle p\rangle$ is a prime ideal, we get that either $x\in\langle p\rangle$ or $k\in\langle p\rangle$. If $x\in \langle p\rangle$ , then $\langle x\rangle=\langle p\rangle$. If $k\in\langle p\rangle$, then $k=py\implies p=pyx\implies p(1-yx)=0\implies yx=1\implies x$ is a unit $\langle x\rangle=R$. Thus $P$ is maximal.
\end{proof}
\begin{lemma}
$\mathbb{R}[x]$ is a principal ideal domain.
\end{lemma}
\begin{proof}
Let $I$ be an ideal of $\mathbb{R}[x]$. There are two cases to consider. In the first we have $I=\{0\}$, which means that $I=0\mathbb{R}[x]$ and hence is a PID. In the second case $I\neq0$. This means that there exists a nonzero polynomial $f\in I$ with minimal degree. Let $g\in I$. By the division algorithm there exists polynomials $q,r\in\mathbb{R}[x]$ such that:
$$g(x)=f(x)q(x)+r(x)$$
where $r(x)=0$ or $0\leq \text{deg }r\leq\text{deg }f$. We rewrite the equation above as:
$$r(x)=g(x)-f(x)q(x)$$
Since $f\in I$ and $q\in\mathbb{R}[x]$ we have by definition of $I$ being an ideal that $fq\in I$. Since $g\in I$ we have that $(g-fq)\in I$. So $r\in I$. But by minimality of the degree of $f$ in $I$ we must have that $r(x)=0$. Therefore:
$$g(x)=f(x)q(x)$$
Since $g\in I$ was arbitrary we have that $I=f(x)\mathbb{R}[x]$. So $I$ is a principal ideal. Hence every ideal in $\mathbb{R}[x]$ is a principal ideal. So $\mathbb{R}[x]$ is a principal ideal domain.
\end{proof}
\begin{theorem}
$\mathbb{R}[x]/(x^2 + 1)\cong\mathbb{C}$
\end{theorem}
\begin{proof}
From lemma 3 $\mathbb{R}[x]$ is a PID and thus from lemma 2 every prime ideal is maximal. From lemma 1 $(x^2+1)$ is irreducible over $\mathbb{R}[x]$, and therefore $(x^2+1)$ is a maximal ideal, so $\mathbb{R}[x]/(x^2+1)$ is a field. Now define the homomorphism $\phi:\mathbb{R}[x]\rightarrow\mathbb{C}$, such that $\phi(f)=f(i)$, this is the polynomial $f$ evaluated at $i$). It is clear that $(x^2+1)\subseteq \text{ker }\phi$, and equality follows since $(x^2+1)$ is maximal. From the 1st isomorphism theorem for rings, we have $\mathbb{R}[x]/(x^2+1)\cong\mathbb{C}$.
\end{proof}
\noindent$\mathbb{R}[x]/(x^2 + x + 1)$ is isomorphic to
$\mathbb{C}$ as $x^2+x+1$ is irreducible, so by the same argument as in the proof of theorem 1 except with the homomorphism being $\phi(f)=f\left(\frac{-1+\sqrt{3}i}{2}\right)$ we get that $\mathbb{R}[x]/(x^2 + x + 1)\cong\mathbb{C}$.
\end{document}