\documentclass{article}
\usepackage[utf8]{inputenc}
\usepackage{amsmath}
\usepackage{mathtools}
\usepackage{amsthm}
\usepackage{graphicx}
\usepackage{tabularx}
\usepackage{mathtools}
\usepackage{mathrsfs}
\usepackage{enumerate}
\usepackage{amssymb}
\usepackage{accents}
\usepackage{commath}
\usepackage{yfonts}
\usepackage{float}
\usepackage{array}

\title{Stat3001 Assignment 1}
\author{Dominic Scocchera}
\date{March 2023}

\newtheorem{theorem}{Theorem}
\newtheorem{corollary}{Corollary}
\newtheorem{lemma}[theorem]{Lemma}

\begin{document}
\maketitle
\section*{Q1}
We have $x_1,...,x_n\stackrel{iid}{\sim}N(\theta,\theta^2)$ so the likelihood function is:
\begin{align*}
L(\tilde{x};\theta)&=\prod_{i=1}^{n}f(\tilde{x};\theta)\\
&=\prod_{i=1}^{n}(2\pi)^{-\frac{1}{2}}(\theta^2)^{-\frac{1}{2}}\exp\left(-\frac{1}{2}\frac{(x_i-\theta)^2}{\theta^2}\right)\\
&=(2\pi)^{-\frac{n}{2}}(\theta)^{-n}\exp\left(-\frac{1}{2}\sum_{i=1}^{n}\frac{(x_i-\theta)^2}{\theta^2}\right)\\
&=(2\pi)^{-\frac{n}{2}}(\theta)^{-n}\exp\left(-\frac{1}{2}\sum_{i=1}^{n}\frac{x_i^2-2x_i\theta+\theta^2}{\theta^2}\right)\\
&=(2\pi)^{-\frac{n}{2}}(\theta)^{-n}\exp\left(-\frac{1}{2\theta^2}\sum_{i=1}^{n}x_i^2+\frac{1}{\theta}\sum_{i=1}^{n}x_i-\frac{n}{2}\right)\\
&=\exp\left(-\frac{1}{2\theta^2}\sum_{i=1}^{n}x_i^2+\frac{1}{\theta}\sum_{i=1}^{n}x_i\right)(2\pi)^{-\frac{n}{2}}(\theta)^{-n}\exp\left(-\frac{n}{2}\right)\\
&=h_1(T(\tilde{x}))h_2(\tilde{x})\\
\end{align*}
\noindent Hence we have by the Fisher-Neyman factorisation theorem:
$$T(\tilde{x})=\left(\sum_{i=1}^{n}x_i,\sum_{i=1}^{n}x_i^2\right)^T$$
Here $T(\tilde{x})$ is the sufficient statistic for $\theta$. 
\newline\newline
We also have:
\begin{align*}
q&=\text{dim}(T(\tilde{x}))=2\\
d&=\text{dim}(\theta)=1
\end{align*}
Noting that we can not factorise any further as it is a polynomial of different quadratic and linear terms so $q=2$. As $q\neq d$ it does not belong to the regular exponential family.
\section*{Q2}
We have $x_1,...,x_n\stackrel{iid}{\sim}N(\theta,\theta)$ so the likelihood function is:
\begin{align*}
L(\tilde{x};\theta)&=\prod_{i=1}^{n}f(\tilde{x};\theta)\\
&=\prod_{i=1}^{n}(2\pi)^{-\frac{1}{2}}(\theta)^{-\frac{1}{2}}\exp\left(-\frac{1}{2}\frac{(x_i-\theta)^2}{\theta}\right)\\
&=(2\pi)^{-\frac{n}{2}}(\theta)^{-\frac{n}{2}}\exp\left(-\frac{1}{2}\sum_{i=1}^{n}\frac{(x_i-\theta)^2}{\theta}\right)\\
&=(2\pi)^{-\frac{n}{2}}(\theta)^{-\frac{n}{2}}\exp\left(-\frac{1}{2}\sum_{i=1}^{n}\frac{x_i^2-2x_i\theta+\theta^2}{\theta}\right)\\
&=(2\pi)^{-\frac{n}{2}}(\theta)^{-\frac{n}{2}}\exp\left(-\frac{1}{2\theta}\sum_{i=1}^{n}x_i^2+\sum_{i=1}^{n}x_i+n\theta\right)\\
\end{align*}
Now taking the log likelihood we get:
\begin{align*}
\log L(\tilde{x};\theta)&=-\frac{n}{2}\left(\log 2\pi +\log\theta\right)-\frac{1}{2\theta}\sum_{i=1}^{n}x_i^2+\sum_{i=1}^{n}x_i+n\theta\\
\end{align*}
For the regular exponential family we require:
$$L(\tilde{x};\theta)=\frac{b(\tilde{x})\exp(c(\theta)^TT(\tilde{x}))}{a(\theta)}$$
And hence:
$$\log L(\tilde{x};\theta)=\log b(\tilde{x})+c(\theta)^TT(\tilde{x})-\log a(\theta)$$
Here we have:
\begin{align*}
a(\theta)&=\frac{n}{2}(\log2\pi+\log\theta+2\theta)\\
b(\tilde{x})&=\sum_{i=1}^{n}x_i\\
c(\theta)&=-\frac{1}{2\theta}\\
T(\tilde{x})&=\sum_{i=1}^{n}x_i^2\\
\end{align*}
Here $T(\tilde{x})$ is the sufficient statistic for $\theta$ and from this we see that this distribution belongs to the regular exponential family (noting that $q=\text{dim}(T(\tilde{x}))=1=\text{dim}(\theta)=d$ and the Jacobian of $c(\theta)$ is $\frac{1}{2\theta^2}$ which obviously of full rank).
\newline\newline
Taking the derivative of the log likelihood equation and setting it to zero  we get:
\begin{align*}
\frac{d}{d\theta} \log L(\tilde{x};\theta)&=-\frac{n}{2\theta}+\frac{1}{2\theta^2}\sum_{i=1}^{n}x_i^2+n\\
&=0\\
\end{align*}
Multiplying through by $2\theta^2$ we get:
$$0=-n\theta+\sum_{i=1}^{n}x_i^2+n\theta^2$$
This is quadratic in $\theta$ so we get:
$$\hat{\theta}=\frac{-1+\sqrt{1+4\frac{1}{n}\sum_{i=1}^nx_i^2}}{2}$$
As the likelihood function is from the regular exponential family we have:
\begin{align*}
\mathbb{E}\left[\sum_{i=1}^nx_i^2\right]&=\sum_{i=1}^n\mathbb{E}\left[x_i^2\right]\\
&=\sum_{i=1}^n\text{Var}(X_i)+\mathbb{E}\left[x_i\right]^2\\
&=\sum_{i=1}^n\theta+\theta^2\\
&=n(\theta+\theta^2)\\
&=\sum_{i=1}^nx_i^2\\
\end{align*}
Rearranging we get:
$$0=n\theta^2+n\theta-\sum_{i=1}^nx_i^2$$
Which is quadratic in $\theta$ so we get:
\begin{align*}
\hat{\theta}&=\frac{-n+\sqrt{n^2+2n\sum_{i=1}^nx_i^2}}{2n}\\
&=\frac{-1+\sqrt{1+4\frac{1}{n}\sum_{i=1}^nx_i^2}}{2}\\
\end{align*}
As the likelihood is from the regular exponential family the suffecient statistic is also complete which guarentees the estimator of $\theta$ is unique. 
\end{document}